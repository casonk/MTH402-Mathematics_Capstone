\documentclass[a4paper, 12pt]{article} %
\usepackage{graphicx,amssymb} %
\usepackage{url}			       % For \url
\usepackage{xcolor}
\usepackage[left=1cm, top=2cm, bottom = 2cm, right=1cm, nohead, nofoot]{geometry}
\usepackage{hyperref}

%\textwidth=15cm \hoffset=-1.0cm %
%\textheight=25cm \voffset=-1.5cm %

\pagestyle{empty} %

\date{\today} %

\def\keywords#1{\begin{center}{\bf Keywords}\\{#1}\end{center}} %

% Please, do not change any of the above lines

\tolerance=1
\emergencystretch=\maxdimen
\hyphenpenalty=10000
\hbadness=10000

\begin{document}

% Type down your paper title
\title{MTH 402 CAPSTONE - APPLICATIONS IN NETWORKS}

\vspace{0.25cm}
% Authors
\author{Cason Konzer  \\ %
       University of Michigan - Flint \\ % Affiliation 1
       % Add authors and affiliation as needed 
       \textit{ \color{violet}
       \href{mailto:casonk@umich.edu}{casonk@umich.edu}}  % Only one corresponding e-mail
       }%

\maketitle

\thispagestyle{empty}

% The abstract
%\vspace{2.5cm}

\begin{abstract}
\vspace{0.25cm}

Within mathematics, there are certain esteemed historical problems widely taught and know by those studying the field. 
To reminisce for a moment, I was introduced to two of these problems in my discrete mathematics class while in my second semester of undergraduate education. 
The seven bridges of Königsberg and the travelling salesman problem, both problems are prominent in graph theory.
Both of these problems can be easily visualized, land masses connected by bridges and cities by roads, map reading is instilled within common human culture. 
In a similar manner a graph represented by nodes connected by edges is intuitive while scale is small. 
We can represent so much in this format, infrastructure, relationships, information flow, etc. 
Once networks become large, visualization and intuition fall off, and abstracts allow for greater understanding. 

In a brief overview of the study of networks \cite{struc_funct_cnets} Newman focusing on real world network types, properties of networks, null graph models and processes taking place on networks. 
The study of networks in general, has drawn initial attention by mathematicians, but in the recent century, and moreover the recent decades, computer and data scientists alike have been having a field day with their applications. 
Within the even more recent years, the study of (social) media networks has grown rapidly. 
Individuals are being persuaded by bots, Trump is banned from twitter while president, and idealogies are becoming seemingly more radicalized.
Community detection and clustering is an interesting approach to tackle these issues, what if we could identify bot nets, or predict attacks by leveraging media networks?

In 2008 the Louvain algorithm \cite{louvain} for detecting communities within networks was published, and has become increasingly popular due to its advantageous run time. 
Additionally, the algorithm was shortly after extended to directed networks, akin to many media networks, and to include a `resolution' parameter \cite{louvain_resolution} providing the ability to tune for community size/density. 
Additional community detections methods have emerges since such as the Leiden \cite{louvain_2_leiden} and BigCLAM \cite{bigclam} algorithms. 
No matter the method, detecting communities is only an initial step to any research effort in media networks. 
To be of relevance, communities must be labeled in a meaningful way, and further their characteristics must be reviewed. 

This type of study has been of persistent interest within the social sciences. 
In the late 80s Krackhardt and Stern \cite{inet_crises} used leveraged their university position to study group dynamics of organizations via student assignments. 
A brief summary of their results finds that organizations which extend their reach externally performed superiorly. 
The so called `EI-Index' was developed as a key metric measuring the relationship between internal and external links. 
`Echochambers' and `filter bubbles' have become common words within the literature, and studies on media networks such as Twitter, Facebook, and Reddit \cite{eko_weko, sm_ece, ekoin_ekobtwin} are now commonplace. 
A thorough understanding of the structure of networks, community detection algorithms, and community analysis make for a optimistic outlook on solving societal problems stemming from media networks.


\vspace{0.25cm}

\end{abstract}

\keywords{Communities, Media, Networks} % Write down at least 3 Keywords

\bibliography{onepager.bib}{}
\bibliographystyle{plain}
\end{document}